\chapter{Cosa è un algoritmo}\label{ch:descrizione}
\section{Le regole di un algoritmo}
\begin{enumerate}
    \item Un algoritmo è di lunghezza finita.
    
    \item Esiste un agente di calcolo (la macchina calcolatrice, appunto) che sviluppa la computazione eseguendo le istruzioni dell’ algoritmo.
    
    \item L’agente di calcolo ha a disposizione una memoria, dove vengono immagazzinati i risultati intermedi del calcolo.
    
    \item Il calcolo avviene per passi discreti.
    
    \item Il calcolo non è probabilistico.
    
    \item Non deve esserci alcun limite finito alla lunghezza dei dati di ingresso.
    
    \item  Non deve esserci alcun limite alla quantità di memoria disponibile.
    
    \item Deve esserci un limite finito alla complessità delle istruzioni eseguibili dal dispositivo.
    
    \item Il numero di passi della computazione è finito ma non limitato.
    
    \item Sono ammesse computazioni senza fine.
\end{enumerate}

\section{Linguaggi Formali}
Ogni informazione può essere rappresentata da una sequenza finita di simboli su un opportuno alfabeto finito.\\\\
\textbf{Esempio.}\\La Divina Commedia, il genoma umano, i numeri interi e quelli razionali (per es. in base 10), grafi, polinomi, ecc... Possiamo quindi assumere che i processi di calcolo riguardino stringhe di simboli scelti in un insieme finito e non vuoto.

\subsection{Parole}
\textbf{Definizione}\\
Chiameremo alfabeto un insieme finito non vuoto A di simboli. I suoi elementi sono detti lettere.\\\\
\textbf{Esempi:}\\
$\{a, b\} , \{0, 1\} , \{a, b, c\} , \{0, 1, 2, 3, 4, 5, 6, 7, 8, 9, A, B, C, D, E,F\}$.\\\\
\textbf{Definizione}\\
Ogni sequenza finita di lettere di A si dice parola sull’alfabeto A. L’insieme delle parole sull’alfabeto A sarà denotato con A*.\\\\
\textbf{Esempi}\\
a, abb, ababbabb sono parole sull’alfabeto \{a, b\}.\\
01001010, 0110, 0000 sono parole sull’alfabeto \{0, 1\}.\\\\
Possiamo anche considerare la sequenza priva di lettere, che si denota con $\wedge$ (oppure $\epsilon$) e si dice parola vuota.

\newpage
\subsection{Lunghezza}
Una parola sull’alfabeto A è una sequenza u = a1a2...ak con k $>=$ 0, a1, a2, . . . , ak $\in$ A.\\\\
\textbf{Definizione}\\
\vspace{0.3cm}
L’intero k si dice lunghezza della parola u e si denota con l(u).\\\\
\textbf{Esempi}\\
l(a) = 1, l(aab) = 3, l(ababbabb) = 8, l($\wedge$) = 0.
\subsection{Concatenzione, fattori}
\textbf{Definizione}\\
Si considerino le parole u = a1a2...ak e v = b1b2...bh con k, h $>= 0$ \\
a1, a2, . . . , ak , b1, b2, ... ,bh $\in$ A.\\
La concatenazione di u e v è \textbf{la parola uv}:\\
\begin{center}
uv = a1a2...ak b1b2 ... bh.    
\end{center}
\textbf{Esempi}\\
La concatenazione delle parole abb e aaab è la parola abbaaab.\\
La concatenazione delle parole aaab e abb è la parola aaababb.\\
La concatenazione delle parole baa e $\wedge$ è la parola baa.\\\\ 
\textbf{Definizione}\\
Diremo che una parola v è un \textbf{fattore} di una parola w se risulta w = xvy.\\
Per opportune parole x , y. Nel caso in cui x = $\epsilon$ (risp., y = $\wedge$) il fattore v si dice 
prefisso (\textbf{risp}., \textbf{suffisso}) di w.\\ 
Diremo che v è un fattore proprio se v $\neq$ w.\\\\
\textbf{Esempio}\\
I fattori di abb sono $\wedge$, a, b, ab, bb e abb. Dove i prefissi sono $\wedge$, a, ab e abb ed i suffissi sono  $\wedge$, b, bb e abb.
\subsection{Linguaggio Formale}
\textbf{Definizione}\\
Ogni sottoinsieme di A* si dice linguaggio formale o linguaggio o problema sull’alfabeto A.\\\\
\textbf{Esempio}\\
Sono linguaggi formali sull’alfabeto A = \{a, b\}:\\
L0 = \{a, b\} , L1 = \{a, ab, abb\} , L2 = \{abna $|$ n $>=$ 0\} , L3 = $\emptyset$ , L4 = A*,\\L5 = \{a$^p$ $|$ p primo\}\\
Le espansioni binarie dei numeri primi costituiscono un linguaggio sull’alfabeto \{0,1\}.\\
L’insieme dei polinomi Diofantei nelle variabili x , $x'$, $x''$, $x'''$, . . . che ammettono una radice intera possono essere rappresentati come un linguaggio sull’alfabeto
\begin{center}
\{0, 1, 2, 3, 4, 5, 6, 7, 8, 9, $+$, $-$ ,$^\wedge$, x ,$"$\}
\end{center}

\subsection{Ordinamento}
\textbf{Definizione}\\
Sia A un alfabeto totalmente ordinato. L’ordine radicale (o militare) su A* è definito come segue:\\ 
siano u, v $\in$ A*.\\
Si ha u $<$ v se è soddisfatta una delle due condizioni seguenti:
\begin{itemize}
    \item l(u) $<$ l(v)
    
    \item l(u) = l(v) e u precede v nell’ordine lessicografico 
    
    (cioè u = ras, v = rbt con r,s,t $\in$ A* a, b $\in$ A e a $<$ b nell’ordinamento dell’alfabeto A)
\end{itemize}
\textbf{Esempio}\\
I numeri naturali sono rappresentati in base 10 da parole sull’alfabeto:
\begin{center}
    \{0, 1, 2, 3, 4, 5, 6, 7, 8, 9\}, prive di 0 iniziali.
\end{center}
L’ordinamento usuale dei numeri naturali coincide con ordine radicale delle corrispondenti espansioni decimali.
