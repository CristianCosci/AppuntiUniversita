\chapter{Security Design}

I primi 8 sono stati definiti da un docente di nome
Saelzer in un articolo del 1975. Gli altri sono stati
aggiunti dai ricercatori nel tempo. Nello specifico:

\paragraph{Economy of mechanism}
La progettazione delle misure di sicurezza incorporate
nell'hardware che nel software dovrebbe essere la più semplice e piccola possibile.
Tenere il sistema più semplice possibile nel progetto, nell'implementazione,
per quanto riguarda il
numero di operazioni, le interazioni con altre componenti e addirittura nella
specifica. “Semplice”
significa che ci sono meno cose da controllare e quindi in caso di errori è più
facile identificarli e
correggerli.
Particolare attenzione deve essere posta nelle interfacce ad altri moduli
(possono fare assunzioni
non verificate); bisogna sempre definire completamente tutte le interfacce,
ad esempio le variabili
ambientali e le variabili globali, nonché le liste dei parametri.
Include altri principi come:

\begin{itemize}
    \item \textbf{KISS} è un acronimo usato in progettazione, che sta per
          “\textit{Keep It Simple, Stupid}”, ossia
          "\textit{rimani sul semplice, stupido}".
          In riferimento al codice sorgente di un programma significa
          non occuparsi delle ottimizzazioni fin dall'inizio, ma cercare invece
          di mantenere uno stile di
          programmazione semplice e lineare, affidando le ottimizzazioni al
          compilatore o a
          successive fasi dello sviluppo;
    \item \textbf{Minimalists Design}: gli sviluppatori possono creare interfacce
          utente fatte per essere il più
          semplice possibile eliminando pulsanti e finestre di dialogo che
          possono potenzialmente
          confondere l'utente;
    \item \textbf{Worse is better}: la qualità non è necessariamente funzionale;
    \item \textbf{YAGNI}: l'espressione “\textit{you aren't gonna need it}”,
          dall'inglese "\textit{non ne avrai bisogno}",
          in ingegneria del software si riferisce a un principio secondo cui un
          programmatore non dovrebbe sviluppare software che implementi funzionalità non
          esplicitamente richieste.\footnote{Un esempio è la vulnerabilità
              trovata all'interno della libreria Log4J. Per saperne di più
              \url{https://www.cybersecurity360.it/nuove-minacce/log4shell-ecco-come-funziona-lexploit-della-vulnerabilita-di-log4j/},
              \url{https://logging.apache.org/log4j/2.x/security.html}.}
\end{itemize}

\paragraph{Fail-safe defaults}
Le decisioni di accesso dovrebbero essere basate sull'autorizzazione
piuttosto che sull'esclusione: la situazione predefinita è la mancanza di accesso
e il sistema
di protezione identifica le condizioni in cui l'accesso è consentito
(default deny).
Per default l'accesso di un soggetto ad un oggetto deve essere negato.
Ogni diritto deve essere concesso esplicitamente.
Inoltre se un'azione fallisce il sistema deve restare sicuro, come se
l'azione non fosse stata svolta.
In realtà sarebbe più corretto parlare di \textbf{fail-secure}: un sistema
è fail-secure quando in caso di un
determinato guasto, risponde in modo tale che l'accesso o i dati vengano negati.
Un sistema \textbf{fail-safe}, invece, in caso di guasto non provoca alcun danno
o solo in minima parte.
Mentre i prodotti fail-safe sono sbloccati quando viene tolta la corrente
(cioè viene applicata la corrente per bloccare la porta), i prodotti
fail-secure sono bloccati quando viene tolta la corrente
(cioè viene applicata la corrente per sbloccare la porta)
In sintesi i sistemi:

\begin{itemize}
    \item \textbf{Fail-secure}: vige la default-deny nei firewall.
          I programmatori dovrebbero controllare i valori
          di ritorno in caso di eccezioni o fallimenti; si devono basare le
          decisioni sul permesso, non
          tanto sull'esclusione.
    \item \textbf{Fail-insecure}: rappresentano esempi di design sbagliati.
\end{itemize}

\paragraph{Complete mediation}
Bisogna controllare ogni accesso: di solito il controllo viene fatto al primo.
Ad esempio in UNIX il
controllo di accesso in lettura ad un file è fatto solo all'apertura del file e
non ad ogni operazioni di lettura.
Se i permessi cambiano durante l'operazione, si possono generare accessi non
validi.

\paragraph{Open design}
La progettazione di un meccanismo di sicurezza dovrebbe essere aperta e
non segreta.
La sicurezza non dovrebbe dipendere dalla segretezza nella progettazione o
dell'implementazione del sistema, ma dalla bontà di tale progettazione ed
implementazione. Si potrebbe però intendere
che il codice sorgente dovrebbe essere pubblico.
È molto importante ricorda che questo metodo non va bene con informazioni
tipo passwords o chiavi crittografiche.

\paragraph{Separation of privilege}
Sono necessari molteplici attributi di privilegio per ottenere
l'accesso a una risorsa limitata.
Non si può concedere un diritto in base ad una condizione sola.
Ad esempio in molti Unix per
diventare root bisogna conoscere la password di root e far parte del gruppo wheel.
Perciò si devono richiedere più condizioni per garantire un privilegio.
Anche il design può dipendere dai diversi privilegi: non sempre l'accesso al
file viene considerato
un privilegio, come lo stesso accesso alla rete.
Vanno inoltre distinti lettura e scrittura.
Vige anche il principio

\begin{itemize}
    \item della “separazione funzionale”: persone diverse, pur avendo diversi
          compiti, devono cooperare;
    \item del “controllo duale”.
\end{itemize}

\paragraph{Least privilege}
Ogni processo e ogni utente del sistema dovrebbe operare ricorrendo al
minor numero di privilegi necessari per eseguire il compito.
I meccanismi di accesso alle risorse dovrebbero essere condivisi il meno possibile.
L'informazione può fluire attraverso i canali condivisi.

\paragraph{Least common mechanism}
Il progetto dovrebbe ridurre al minimo le funzioni condivise dai
diversi utenti, garantendo la sicurezza reciproca (separazione dei compiti).
Un soggetto deve possedere solo i diritti di cui ha bisogno per svolgere i propri
compiti. I diritti sono concessi in base alla funzione, non all'identità (RBAC).
I diritti vengono aggiunti quando servono e quando non servono più sono revocati
(sessions/Dynamic).
Non tutti i sistemi, a causa della granularità dei permessi, sono in grado di
soddisfare tale principio.

\paragraph{Psychological acceptability}
Implica che i meccanismi di sicurezza non devono interferire
sul lavoro degli utenti, soddisfacendo al tempo stesso le esigenze di chi
autorizza l'accesso;

\paragraph{Isolation}

\paragraph{Encapsulation}

\paragraph{Modularity}

\paragraph{Layering}

\paragraph{Least astonishment}
