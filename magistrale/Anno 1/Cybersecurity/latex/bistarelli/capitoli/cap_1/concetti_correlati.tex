\section{Concetti Correlati}

\subsection{Assurance}

Il concetto di “\textit{Garanzia}” non è formalmente legato al sistema su cui si
effettua un’analisi di sicurezza informatica. Spesso, infatti, una minaccia
alla sicurezza è dovuta dall'utenza stessa del sistema considerato.
Pensiamo per esempio ad un impiegato negligente che involontariamente permette ad
un virus di infettare il proprio computer, dopo aver inserito una USB.
La sicurezza informatica deve tener conto anche di queste situazioni che
potrebbero di fatto rendere inutili gli sforzi effettuati dall'organizzazione
per evitare che minacce esterne compromettano il proprio sistema informatico.
Un comportamento sbagliato da parte del personale infatti è spesso più rischioso
rispetto ai tentativi di intrusione da parte di attaccanti esterni all’organizzazione.
La natura umana non permette di creare dei protocolli di sicurezza in grado di
difendere i propri sistemi da cattive abitudini o da un atteggiamento noncurante
degli operatori. Solitamente ogni organizzazione utilizza delle policy aziendali
che stabiliscono il comportamento del proprio personale. Una policy non
garantisce una protezione totale, ma permette di ridurre il numero di comportamenti
sbagliati e di conseguenza rende più difficile gli attacchi interni.

\subsection{Minacce}

Una minaccia (\textbf{threat}) è una potenziale violazione della sicurezza.
In realtà non è necessario che avvenga concretamente una violazione per essere
considerata una minaccia basta il solo fatto che ci sia la possibilità che
il sistema debba essere protetto.
Le azioni che causano una minaccia sono chiamati \textbf{attacchi}. Coloro che
mettono in pratica queste azioni, o permettono che esse siano eseguite, sono
invece chiamati \textbf{attaccanti}.
Le \textit{minacce} possono essere classificate in:

\begin{itemize}
      \item \textit{Disclosure}: accesso non autorizzato alle informazioni
            (violazione della riservatezza);
      \item \textit{Deception}: accettazione di dati falsi
            (violazione dell’integrità e dell’autenticazione);
      \item \textit{Disruption}: corrisponde al DoS
            (violazione della disponibilità);
      \item \textit{Usurpation}: controllo non autorizzato di un sistema o parte
            di esso.
\end{itemize}

Tra gli \textit{attacchi} più diffusi troviamo:

\begin{itemize}
      \item \textbf{Snooping}: È una minaccia di tipo \textit{Disclosure}.
            Consiste nell’ascolto o nella lettura passiva (intercettazione) di
            informazioni. Si parla di \textit{Wiretapping} passivo se l’oggetto
            di ascolto è la rete stessa. La riservatezza si occupa di evitare
            che questa minaccia possa essere applicata;
      \item \textbf{Modification}: Consiste nella modifica non autorizzata delle
            informazioni. Copre tre classi di minacce. L’obiettivo potrebbe essere
            una \textit{Deception} se l’intenzione è fare in modo che i dati siano
            accettati in modo alterato. Potrebbe essere una \textit{Disruption}
            ed una \textit{Usurpation} se la modifica dei dati consente l’esecuzione
            di azioni non legittime. Il \textit{Wiretapping} attivo se i dati che
            transitano nella rete sono soggetti ad una modifica non autorizzata.
            Un esempio di attacco è il “\textbf{man-in-the-middle}”, nel quale un
            intruso legge i messaggi dal mittente e li invia modificati al
            destinatario senza farsi notare. L’integrità si occupa di evitare che
            questa minaccia possa essere applicata;
      \item \textbf{Masquerading o Spoofing}: Si tratta di un furto di identità.
            È un attacco di tipo \textit{Deception} e \textit{Usurpation}.
            Si basa sul far credere ad una vittima che l’entità con cui sta
            comunicando è un’altra. L’integrità si occupa di evitare che questa
            minaccia possa essere applicata;
      \item \textbf{Ripudio dell’origine}: Si tratta di una forma di \textit{Deception}
            che consiste nel negare l’origine (invio o creazione) di un determinato
            elemento (ad esempio di un messaggio). L’integrità si occupa di questo
            attacco;
      \item \textbf{Ripudio della ricezione}: Si tratta di una forma di \textit{Deception}
            che consiste nel negare di aver ricevuto un determinato elemento.
            L’integrità si occupa di questo attacco;
      \item \textbf{Denial of Service}: Consiste in una negazione a lungo termine
            di un servizio. È una forma di \textit{Usurpation}, sebbene possa
            rientrare anche nella \textit{Disruption}. L’attaccante non permette
            ad un server di fornire un servizio.
\end{itemize}

\subsection{Policy e Meccanismi}

Le \textbf{Policy} di sicurezza identificano le minacce e definiscono quali sono le
assunzioni da fare ed i requisiti di cui disporre. Stabiliscono ciò che può o
non può essere fatto e definiscono quindi il livello di sicurezza del sistema.
Esse si basano su delle assunzioni che consistono nel definire degli insiemi di
stati sicuri ed altri insicuri.
I \textbf{Meccanismi} sono i metodi che ci consentono di individuare, prevenire
e ripristinare le minacce. Una volta individuati i rischi in termini di sicurezza
ed i loro effetti, si stabiliscono quali contromisure adottare.
L’affidabilità di un meccanismo di sicurezza richiede diverse assunzioni:

\begin{enumerate}
      \item Ogni meccanismo è stato ideato per implementare una o più richieste
            di una politica di sicurezza;
      \item L’unione di tutti i meccanismi copre tutti gli aspetti della policy
            di sicurezza;
      \item Il meccanismo è stato implementato correttamente;
      \item Il meccanismo è installato e amministrato correttamente.
\end{enumerate}

Un meccanismo di sicurezza tipico è dato dalla verifica dell’identità prima di
cambiare una password.
Esistono meccanismi di:

\begin{itemize}
      \item \textit{Prevenzione}: prevengono attacchi derivati dalla violazione delle
            politiche di sicurezza;
      \item \textit{Individuazione}: individuano attacchi di violazione alle
            politiche di sicurezza;
      \item \textit{Ripristino}: rappresentano le azioni da eseguire a fronte di
            un attacco.
            Le possibili alternative sono:
            \begin{itemize}
                  \item il blocco dell’attacco e la riparazione dei danni causati.
                  \item non si agisce sull'attacco ma si immagazzinano informazioni
                        sull'attaccante;
                  \item ritorsione (retaliation) eseguendo un attacco verso
                        l’attaccante.
            \end{itemize}
\end{itemize}