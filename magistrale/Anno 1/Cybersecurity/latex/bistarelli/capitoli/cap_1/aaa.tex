\section{AAA}

I servizi di \textbf{AAA}\footnote{AAAAAAAAAAAAAAAAAAAAAAAAAAAAAAAAAAAAAAAAAAAH}
rappresentano uno strumento fondamentale per la messa in sicurezza dell’infrastruttura
di rete.
AAA è l'acronimo
di \textbf{Authentication}, \textbf{Authorization} and \textbf{Accounting} e in
generale descrive un protocollo (o una famiglia di protocolli) software che
implementa le funzionalità di \textbf{Autenticazione}, \textbf{Autorizzazione}
e \textbf{Tracciabilità} degli utenti: ad esempio all'interno di una infrastruttura
di rete aziendale. Questi elementi sono fondamentali per poter controllare
l'accesso alle risorse ed a eventuali dati confidenziali.

\paragraph{Authentication.}
L’autenticazione (o non ripudio, \textit{non-repudiation}) ha lo scopo di
riconoscere l’utente. Normalmente si fa ricorso ad uno \textit{username}, in
genere pubblico, e di una \textit{password}, che invece è segreta. Per garantire
sicurezza e segretezza, la password deve rispondere a specifiche policy rinnovate
periodicamente.

\paragraph{Authorization.}
L’autorizzazione, seppur integrata con l’autenticazione, è un processo molto
differente. Una volta riconosciuto l’utente, bisogna accertarsi che questo possa
accedere alle sole risorse che gli competono. Le informazioni necessarie per
questo processo sono strettamente legate all’applicativo e alle prestazioni dello
stesso: ogni volta che si desidera accedere ad un dato occorre verificare
l’autorizzazione dell’utente.

\paragraph{Accounting.}
L’accounting (tracciabilità) consente di tracciare l’utilizzo delle risorse.
È fondamentale per scoprire se un account ha compiuto accessi in orari o da luoghi
che risultano insoliti o per verificare azioni sospette,
come un accesso da una rete pubblica mentre l’utente risulta, dalla timbratura cartellini,
ancora presente in azienda.