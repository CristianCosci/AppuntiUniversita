\section{Tipi di Autenticazione}

L'\textbf{autenticazione} è il processo attraverso il quale viene verificata
l'identità di un utente che vuole
accedere ad un computer o ad una una rete. È il sistema che verifica,
effettivamente, che un
individuo è chi sostiene di essere. Stabilisce l'identità di una parte ad un'altra.
Le parti possono essere utenti o computer:

\begin{itemize}
      \item \textbf{computer-computer} (stampa in rete, delega,…)
      \item \textbf{utente-utente} (protocolli di sicurezza, …)
      \item \textbf{computer-utente} (autenticare un server web,…)
      \item \textbf{utente-computer} (per accedere a un sistema…)
\end{itemize}

In realtà sono spesso richieste varie combinazioni di queste.
L'autenticazione è una proprietà primaria, viene richiesta da un corretto
controllo d'accesso.
Garantire l'autenticazione significa fare in modo che un sistema sia in grado di
associare con
certezza un'identità ad una persona.\\

Esistono vari tipi di autenticazione:

\begin{itemize}
      \item \textbf{Locale}: funziona anche offline ed è la singola macchina
            che autentica l'utente (desktop systems);
      \item \textbf{Diretta}: l'autenticazione avviene su una macchina diversa
            da quella dove è avvenuto il
            collegamento; sul server digitando le proprie credenziali
            (per aprire file, fare login..);
      \item \textbf{Indiretta}: l'autenticazione non sulla macchina ma su un server
            di logging remoto (Windows
            domain, radius, kerberos, nis). Un esempio è l'autenticazione che
            facciamo in laboratorio. L'utente richiede al server l'accesso ad una
            risorsa; il server invia la richiesta di accesso al
            programma gestore degli accessi, il quale ritorna l'esito
            dell'autenticazione al server. Quest'ultimo
            infine rigira l'esito all'utente iniziale;
      \item \textbf{Offline}: controllo con chiave pubblica e privata.
            Grazie al certificato associato alla chiave
            pubblica, siamo ricondotti al destinatario (PKI…).
\end{itemize}

\subsection{Autenticazione utente-computer}

L’autenticazione tra un computer ed un utente solitamente avviene tramite
l’utilizzo di \textbf{certificati} (come nel caso bancario); quella tra l’utente
ed un computer avviene sulla base di qualcosa che l’utente:

\begin{itemize}
      \item \textit{Conosce}: informazioni quali password, PIN, ecc.
      \item \textit{Possiede}: cose fisiche o elettroniche che solo l’utente ha,
            come chiavi convenzionali, carte magnetiche o smart, ecc.
      \item \textit{È}: caratteristiche biometriche quali
            impronte digitali, l’iride, tono di voce, ecc
\end{itemize}

\subsection{Autenticazione su conoscenza}

Questo metodo si basa sulla conoscenza di una coppia di elementi, \textbf{userID}
e \textbf{password}, che forniscono una prova dell’identità.
Esiste un database locale dove si ricerca la coppia di informazioni:
se viene trovata una corrispondenza, l’utente viene riconosciuto.
Questo metodo è \textit{antico} ma \textit{estremamente diffuso} per via della sua
economicità e \textit{semplicità} di gestione ed implementazione.
Risulta però essere debole e poco resistente una volta scoperta una password,
in quanto la vulnerabilità agli attacchi sarà molto alta.

\subsection{Gestione delle password}

L’utente fornisce userID e password; questi vengono ricercati nel database e se
sono presenti è consentito l’accesso.
Questo sistema però presenta alcune debolezze. In primo luogo, nel database
userID e password sono in chiaro: poteva capitare che queste informazioni
venissero utilizzate per entrare illecitamente. Poteva anche presentarsi il
rischio che il database stesso venisse rubato, è vero che il file era protetto,
ma il suo contenuto era in chiaro, per cui non vi era alcuna protezione qualora
qualcuno se ne impossessasse. Per questo motivo dal 1960 dal MIT sono state
adottate delle strategie per migliorare tale implementazione. Le password,
infatti, non venivano più ricercate nel database ma venivano memorizzate in
chiaro su un file di sistema in RAM, protetto da politica di sicurezza.
In un secondo momento il file veniva letto dal database e veniva effettuato
il controllo.
Dal 1967, ad opera di Cambridge, è stata introdotta la criptazione delle password
al momento del loro inserimento tramite una specifica funzione, per cui venivano
memorizzati solo gli hash e non le password stesse.
Il vantaggio stava nel fatto che le informazioni nel database non fossero più in
chiaro e non vi era più la necessità di utilizzare il file memorizzato in RAM.
Allo stesso tempo però il file degli hash poteva essere aperto in lettura da
chiunque e per due password uguali si aveva il medesimo hash.
Il problema delle password uguali viene risolto con l’aggiunta del \textit{sale}
(\textbf{salt}), ossia una sequenza di bit generata dal sistema. La password,
quindi, risulta diversa dalla sua omonima grazie a questa informazione casuale
in più.
Nel file viene memorizzato non solo la password, ma anche il sale corrispondente.
Questo viene posizionato in una directory nascosta (\textbf{shadow}), in nessun
modo accessibile da alcun utente ad eccezione di root.
Per quanto riguarda l’autenticazione, l’utente inserisce la password che conosce:
dal file delle password viene preso il sale dalla riga che corrisponde all’userID
e si codificano le due informazioni, \(salt + password\). Se il risultato è
uguale alla riga memorizzata nel database, allora l’utente può entrare.
Anche questo file viene aperto in lettura da tutti.
Tale problema verrà risolto solo nel 1987.

\subsubsection{Vulnerabilità delle password}
Di seguito un elenco delle vulnerabilità più comuni a cui sono soggette le
password:

\begin{itemize}
      \item \textbf{Guessing}: indovinate. Mai usare le singole parole;
      \item \textbf{Snooping}: sbirciate mentre vengono inserite.
            Per questo vengono sostituite dagli asterischi
            (Shoulder surfing);
      \item \textbf{Sniffing}: intercettate durante la trasmissione in rete
            (Keystroke sniffing);
      \item \textbf{Spoofing}\footnote{Ricordarsi che il termine corretto è
                  \textit{Spoofing} e non \textit{Spooping}, qui lo SCAT non è bene
                  accetto \emoji{poop}}: acquisite da terze parti che impersonano
            l’interfaccia di login (Trojan login);

\end{itemize}

\subsubsection{Come difendersi ?}

Ecco alcune contromisure utili per difendersi e prevenire i principali attacchi
alle password:

\begin{itemize}
      \item \textit{Guessing attack}:
            utilizzare un \textbf{Audit-log} ed introdurre un
            \textbf{limite massimo} per il numero di sbagli;
      \item \textit{Social engineering}: il cambio password è abilitato solo in
            specifiche circostanze.
            Può anche esserci una policy dedicata;
      \item \textit{Shoulder surfing}: si utilizza la tecnica del
            \textbf{Password Blinding}: vedo gli asterischi al posto della
            password oppure nessun carattere;
      \item \textit{Keystroke sniffing}; utilizzo tecniche di
            \textbf{Memory Protection};
      \item \textit{Trojan login}: chiave speciale per fare login;
      \item \textit{Offline dictionary attack}: utilizzo uno
            \textbf{Shadow Password} per evitare che
            il file delle pw venga rubato.
\end{itemize}

\subsection{Autenticazione su possesso}

In questo caso il \textbf{possesso} di un \textbf{token} fornisce la prova per
il riconoscimento.
Chi possiede un oggetto come una carta magnetica, una smart card o smart token,
è autorizzato a compiere determinate azioni. Tale tipo di autenticazione
solitamente non è nominale, ma può essere accompagnata dal riconoscimento
dell’user. Rubando il token, infatti, non si fa altro che impersonare l’utente.
Ogni token memorizza una chiave che deve essere inserita per sbloccarlo.
Nel caso del bancomat, per esempio, viene richiesto un PIN per utilizzare la carta.
Il vantaggio di questo strumento è che è molto complesso estrarre
un’informazione dal token.

\paragraph{Tipi di Token: }

\begin{itemize}
      \item Carte magnetiche, ormai obsolete;
      \item Smart card per memorizzare una pwd robusta:
            \begin{itemize}
                  \item Memory card: ha una memoria ma non ha capacità computazionali.
                        È impossibile
                        controllare o codificare il PIN, ma essendo trasmesso in chiaro può
                        essere soggetto
                        a sniffing;
                  \item Microprocessor card: sono più evoluti. Ha una memoria e
                        un microprocessore, ma
                        può esserci un controllo o la codifica del PIN;
            \end{itemize}
      \item Smart token:
            \begin{itemize}
                  \item Protetto da PIN;
                  \item  Microprocessor card + tastierino e display
                  \item  Vero e proprio computer;
                  \item  Svantaggi: costoso e fragile
                  \item  Funzionamento: Hanno una Chiave segreta (seme o seed)
                        memorizzata dalla fabbrica,
                        condivisa col server. Preleva le informazioni esterne,
                        per esempio il PIN inserito oppure
                        l'ora, per generare una one-time password.
                        La Password compare poi sul display e viene
                        rinnovata ogni 30-90 secondi.
                        La Sincronizzazione col server avviene grazie al seme ed un
                        algoritmo comune;
            \end{itemize}
\end{itemize}

\section{Autenticazione su caratteristiche}

Il possesso da parte dell’utente di alcune caratteristiche univoche fornisce
prova della sua identità.
Le caratteristiche possono essere:

\begin{itemize}
      \item \textbf{Fisiche}: impronte digitali, forma della mano, impronta
            della retina o del viso, ecc.
      \item \textbf{Comportamentali}:
            firma in riferimento alle sue caratteristiche quali pressione
            della penna o inclinazione delle lettere, timbro della voce,
            velocità nella scrittura, ecc.
\end{itemize}

Lo schema di funzionamento di tale tecnica, prevede una fase iniziale di
\textbf{campionamento} in cui vengono eseguite più misurazioni sulla
caratteristica d’interesse, in modo da stabilire un margine d’errore e viene
definito un modello (\textbf{template}).
Durante la fase di \textbf{autenticazione}, viene confrontata la caratteristica
appena misurata rispetto al template: c’è successo se i due valori corrispondono
a meno di una \textit{tolleranza}, che va definita attentamente.
Se è troppo alta consento l'accesso ad altri utenti, se è troppo bassa potrei
non consentire l'accesso neanche all'utente a cui magari appartiene la
caratteristica misurata.
Ottenere una perfetta uguaglianza è tecnicamente un’operazione impossibile.
Il problema fondamentale di questa tecnica sta nell’identificare il giusto
template da associare all’individuo in analisi.
Va fatto notare che l’autenticazione biometrica è sempre probabilistica,
ovvero dipende da come è settato il parametro soglia: se la soglia
è \textbf{molto bassa} risulterà difficile che l’intruso si autentichi al posto
dell’utente,
ma il numero di falsi negativi sarà alto; se invece la soglia
è \textbf{molto alta}, non avrò mai falsi negativi in quanto l'utente reale
verrà sempre autenticato ma potrebbero esserci falsi positivi da parte di un
attaccante.

\paragraph{Osservazioni:}
è necessario scegliere un tipo di autenticazione
biometrica adatta alla circostanza. Occorre che il sistema non sia troppo
invasivo, poiché ciò potrebbe non essere accettato dall'utente.
Inoltre, non possono essere utilizzate informazioni che metterebbero a
repentaglio la sua privacy. Per esempio, l’autenticazione tramite retina non
viene quasi mai impiegata in quanto da essa si potrebbe rinvenire alla presenza
di alcune malattie. L'impronta digitale rimane il meccanismo più usato,
anche se è il meno pratico ed efficace.

%% No esempio