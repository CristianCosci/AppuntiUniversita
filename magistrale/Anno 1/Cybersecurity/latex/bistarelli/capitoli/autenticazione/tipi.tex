\section{Tipi di Autenticazione}

L'\textbf{autenticazione} è il processo attraverso il quale viene verificata
l'identità di un utente che vuole
accedere ad un computer o ad una una rete. È il sistema che verifica,
effettivamente, che un
individuo è chi sostiene di essere. Stabilisce l'identità di una parte ad un'altra.
Le parti possono essere utenti o computer:

\begin{itemize}
      \item \textbf{computer-computer} (stampa in rete, delega,…)
      \item \textbf{utente-utente} (protocolli di sicurezza, …)
      \item \textbf{computer-utente} (autenticare un server web,…)
      \item \textbf{utente-computer} (per accedere a un sistema…)
\end{itemize}

In realtà sono spesso richieste varie combinazioni di queste.
L'autenticazione è una proprietà primaria, viene richiesta da un corretto
controllo d'accesso.
Garantire l'autenticazione significa fare in modo che un sistema sia in grado di
associare con
certezza un'identità ad una persona.\\

Esistono vari tipi di autenticazione:

\begin{itemize}
      \item \textbf{Locale}: funziona anche offline ed è la singola macchina
            che autentica l'utente (desktop systems);
      \item \textbf{Diretta}: l'autenticazione avviene su una macchina diversa
            da quella dove è avvenuto il
            collegamento; sul server digitando le proprie credenziali
            (per aprire file, fare login..);
      \item \textbf{Indiretta}: l'autenticazione non sulla macchina ma su un server
            di logging remoto (Windows
            domain, radius, kerberos, nis). Un esempio è l'autenticazione che
            facciamo in laboratorio. L'utente richiede al server l'accesso ad una
            risorsa; il server invia la richiesta di accesso al
            programma gestore degli accessi, il quale ritorna l'esito
            dell'autenticazione al server. Quest'ultimo
            infine rigira l'esito all'utente iniziale;
      \item \textbf{Offline}: controllo con chiave pubblica e privata.
            Grazie al certificato associato alla chiave
            pubblica, siamo ricondotti al destinatario (PKI…).
\end{itemize}

\subsection{Autenticazione utente-computer}

\subsection{Autenticazione su conoscenza}

\subsection{Gestione delle password}

\subsubsection{Vulnerabilità delle password}

\subsubsection{Come difendersi dagli attacchi alle pw}

\subsection{Autenticazione su possesso}

\paragraph{Tipi di Token: }

\begin{itemize}
      \item Carte magnetiche, ormai obsolete;
      \item Smart card per memorizzare una pwd robusta:
            \begin{itemize}
                  \item Memory card: ha una memoria ma non ha capacità computazionali.
                        È impossibile
                        controllare o codificare il PIN, ma essendo trasmesso in chiaro può
                        essere soggetto
                        a sniffing;
                  \item Microprocessor card: sono più evoluti. Ha una memoria e
                        un microprocessore, ma
                        può esserci un controllo o la codifica del PIN;
            \end{itemize}
      \item Smart token:
            \begin{itemize}
                  \item Protetto da PIN;
                  \item  Microprocessor card + tastierina e display
                  \item  Vero e proprio computer;
                  \item  Svantaggi: costoso e fragile
                  \item  Funzionamento: Hanno una Chiave segreta (seme o seed)
                        memorizzata dalla fabbrica,
                        condivisa col server. Preleva le informazioni esterne,
                        per esempio il PIN inserito oppure
                        l'ora, per generare una one-time password.
                        La Password compare poi sul display e viene
                        rinnovata ogni 30-90 secondi.
                        La Sincronizzazione col server avviene grazie al seme ed un
                        algoritmo comune;
            \end{itemize}
\end{itemize}

\section{Autenticazione su caratteristiche}

%% No esempio