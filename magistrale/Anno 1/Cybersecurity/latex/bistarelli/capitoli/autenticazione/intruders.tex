\section{Intruder}

Abbiamo tre classi di intruders (intrusi):

\begin{itemize}
      \item \textbf{Masquerader}: Un individuo che non è autorizzato a utilizzare il
            computer e che penetra nei
            controlli di accesso di un sistema per sfruttare l'account di un utente
            legittimo;
      \item \textbf{Misfeasor}\footnote{Ricordarsi che non è
                  il nome di un pokemon !}: Un utente legittimo che accede a dati,
            programmi o
            risorse per i
            quali tale
            accesso non è autorizzato o che è autorizzato a tale accesso ma
            abusa dei suoi
            privilegi;
      \item \textbf{Clandestine User}: Una persona che prende il controllo di supervisione
            per eludere l'auditing
            e i controlli di accesso o sopprimere la raccolta di audit;
\end{itemize}

\paragraph{Intrusion Detection System: }
L'Intrusion Detection System o \textbf{IDS} è un dispositivo software o hardware
utilizzato per
identificare accessi non autorizzati ai computer o alle reti locali.
Le intrusioni rilevate possono
essere quelle prodotte da cracker esperti, da tool automatici o da utenti
inesperti che utilizzano
programmi semiautomatici.

Lo scopo generale di un IDS è informare che potrebbe esserci un'intrusione nella rete. Gli avvisi
includono generalmente informazioni sull'indirizzo di origine dell'intrusione,
l'indirizzo di
destinazione/vittima e il tipo di attacco sospetto”.
Due sono le categorie base: sistemi basati sulle firme (signature) e sistemi
basati sulle anomalie (anomaly).
La tecnica basata sulle firme è in qualche modo analoga a quella per
il rilevamento dei
virus (adopera il machine learning), essa permette di bloccare file infetti e
si tratta di una tra le tecniche più utilizzate. I sistemi basati sul rilevamento delle anomalie utilizzano un
insieme di regole che
permettono di distinguere ciò che è "normale" da ciò che è "anormale".
È importante sapere che un IDS non può bloccare o filtrare i pacchetti in ingresso
ed in uscita, né
tanto meno può modificarli. Un IDS può essere paragonato ad un antifurto mentre
il firewall alla
porta blindata. L'IDS non cerca di bloccare le eventuali intrusioni, cosa che
spetta al firewall, ma
cerca di rilevarle laddove si verifichino.
Per ogni rete è necessario un IDS che agisce solo sulla stessa. Se si volesse
però avere
un'architettura più complessa, si potrebbe includere un gestore centrale:
gli IDS fanno da sensore,
inviano tutto c'è che rilevano nella rete di pertinenza al gestore,
il quale accumula ed analizza tutti i
log ricevuti per capire se gli eventi hanno una qualche correlazione.

\paragraph{Honeypot {\normalfont \emoji{honey-pot}}:}
Un honeypot rappresenta una strategica misura di sicurezza con la quale gli
amministratori
di un server ingannano gli hacker e gli impediscono di colpire.
Un “barattolo di miele” simula i
servizi di rete o programmi per attirare i malintenzionati e proteggere il
sistema da eventuali
attacchi. In pratica gli utenti configurano gli honeypot, utilizzando delle
tecnologie lato server e lato
client. Solitamente la trappola consiste in un computer o un sito che sembra
essere parte della rete
e contenere informazioni preziose, ma che in realtà è ben isolato e non ha
contenuti sensibili o
critici; potrebbe anche essere un file, un record, o un indirizzo IP non utilizzato.
Il valore primario di un honeypot è l'informazione che esso dà sulla natura e la
frequenza di
eventuali attacchi subiti dalla rete.
Gli honeypot possono portare dei rischi ad una rete e devono essere maneggiati
con cura. Se non
sono ben protetti, un attacker potrebbe usarli per entrare in altri sistemi.

Gli honeypot possono essere di due tipi:
\begin{itemize}
      \item \textbf{Bassa Interazione}: sono macchine facili da gestire. Emulano perfettamente
            i servizi, ma l'intruder non riesce a prendere completamente il controllo in
            quanto la macchina è vuota;
      \item \textbf{Alta Interazione}: macchine dove vi è effettivamente un servizio e
            l'utente può utilizzarlo.
            Sono la tipologia migliore perché più realistiche ma sono più complesse
            da gestire;
\end{itemize}