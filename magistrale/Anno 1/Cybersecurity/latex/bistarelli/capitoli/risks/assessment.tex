\section{Risk Assessment}

Il processo di “\textit{Risk Assessment}” è usato per determinare l'ampiezza
delle potenziali minacce ad un sistema IT ed identificare tutte le possibili
contromisure per ridurre o eliminare tali voci di rischio.
Vengono identificati:

\begin{itemize}
    \item \textbf{asset}: qualsiasi bene di proprietà di un'azienda
          (macchinari, merci, ma anche il database o
          la rete) che deve essere protetto;
    \item \textbf{minacce}: possibili attacchi, quindi non ancora avvenuti;
    \item \textbf{vulnerabilità};
    \item \textbf{contromisure}: da implementare per evitare le vulnerabilità;
\end{itemize}

Di conseguenza vengono determinati:

\begin{itemize}
    \item impatto prodotto dalle minacce: per capire quale vulnerabilità
          andare a coprire rispetto ai
          mezzi finanziari a disposizione;
    \item fattibilità delle minacce: probabilità con cui può verificarsi
          un attacco;
    \item complessivo livello di rischio
          (che decido di correre come azienda nel globale).
\end{itemize}
