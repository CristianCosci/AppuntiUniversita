\chapter{IDManag}

\section{intro}
Il \textbf{furto di identità} si verifica quando qualcuno utilizza le informazioni personali di identificazione di un'altra persona, come il nome, il numero di identificazione o il numero di carta di credito, senza il suo permesso, per commettere frodi o altri crimini.

\subsection{Identità digitale}
Un'\textbf{identità digitale} è un'informazione su un'entità utilizzata dai sistemi informatici per rappresentare un agente esterno. Tale agente può essere una persona, un'organizzazione, un'applicazione o un dispositivo.

ISO/IEC 24760-1 definisce l'\textbf{identità} come "insieme di attributi relativi a un'entità".

Le informazioni contenute in un'\textbf{identità digitale} consentono di valutare e autenticare un utente che interagisce con un sistema aziendale sul web/Network, senza il coinvolgimento di operatori umani.

\subsection{Autenticazione}
L'autenticazione è un aspetto fondamentale dell'attribuzione di identità basata sulla fiducia, in quanto fornisce una garanzia codificata dell'identità di un'entità a un'altra.

Le \textbf{metodologie di autenticazione} includono la presentazione di un \textbf{oggetto univoco} come una carta di credito bancaria, le \textbf{informazioni riservate} come una una password o la risposta a una domanda prestabilita, \textbf{la conferma della proprietà di un indirizzo e-mail}, e soluzioni più robuste ma relativamente costose utilizzando metodologie di crittografia.

\subsection{Autorizzazione}

L'autorizzazione è la determinazione di qualsiasi entità che controlla le risorse che l'autenticato può accedere a tali risorse.

L'autorizzazione dipende dall'autenticazione, perché l'autorizzazione richiede la verifica dell'attributo critico.


\section{Identy}

I concetti di \textbf{identità}, \textbf{identificatore} e \textbf{account} sono strettamente correlati ma diversi.

\subsection{Identifier}

Il termine "\textbf{identificatore}" si riferisce a un singolo attributo il cui scopo è quello di identificare in modo univoco una persona o un'entità, all'interno di un contesto specifico. (indirizzo email, numero passaporto ecc...)

\subsection{Identità}

Il termine "identità" è definito come un insieme di attributi associati a una specifica persona/entità in un particolare contesto. Un'identità comprende uno o più identificatori e può contenere altri attributi associati a una persona/entità.

\subsubsection{Attributi}
Le identità umane possono includere attributi come il nome, età, indirizzo, numero di telefono, colore degli occhi e titolo di lavoro.
Le identità non umane possono includere attributi come il proprietario, l'indirizzo IP e forse un numero di modello o di versione. 

Gli attributi che compongono un'identità possono essere utilizzati per l'autenticazione e l'autorizzazione, oltre che per trasmettere informazioni sull'identità alle applicazioni.

Un'identità online consiste in almeno un identificatore e un insieme di attributi per un utente/entità in un particolare contesto, come un'applicazione o una suite di applicazioni. 

\subsection{Account}

Un'identità è associata a un account in ciascuno di questi contesti.

Definiamo un account come un costrutto locale all'interno di una data applicazione o suite di applicazioni che viene utilizzato per eseguire azioni all'interno di quel contesto.

Gli attributi di identità possono essere contenuti account object di un'applicazione, oppure possono essere memorizzati separatamente e referenziati dall'oggetto conto. 

\subsection{Separazione tra ID e account}

Un account può avere un proprio identificativo oltre a quello dell'identità ad esso associata. Avere un identificatore dell' account separato dall'identità associato all'account fornisce un grado di separazione. 

L'identificativo dell' account può essere utilizzato in altri record dell'applicazione per rendere più facile per gli utenti cambiare il nome utente o altro identificatore associato al proprio account.  

Si noti che un account può avere più di un'identità associata ad esso attraverso l'account linking.

\subsection{Non Human Identifier}

Anche gli attori non umani possono certamente avere un'identità.
I componenti software che fungono da agenti o bot e i dispositivi intelligenti possono avere un'identità e possono interagire con altri software o dispositivi in modi che richiedono autenticazione e autorizzazione, proprio come gli attori umani

\subsection{IDM System}

Un sistema di gestione delle identità (IdM) è un insieme di servizi che supportano la creazione, la modifica e la rimozione delle identità e degli account associati, nonché l'autenticazione e l'autorizzazione necessarie per accedere alle risorse.

I sistemi di gestione dell'identità sono utilizzati per proteggere risorse online da accessi non autorizzati e costituiscono un parte importante di un modello di sicurezza completo. 

\section{Eventi in un ciclo di vita di un'identità}
\subsection{Provisioning}
L'atto di creare un account e le relative informazioni di identità  è spesso indicato come provisioning. L'obiettivo della fase di provisioning è quello di stabilire un account con i relativi dati di identità.

Si tratta di ottenere o assegnare un identificativo univoco per l'identità, opzionalmente un identificativo univoco per l'account distinto da quello dell'identità, creare un account e associare gli attributi del profilo dell'identità all'account.

\subsection{AUTHORIZATION}

Quando si crea un account, spesso è necessario specificare cosa può fare l'account, sotto forma di privilegi.

Il termine autorizzazione indica la concessione di privilegi che regolano le attività di un account.

L'autorizzazione di un account viene generalmente effettuata al momento della sua creazione e può essere aggiornata nel tempo. 

\subsection{AUTHENTICATION}

L'utente fornisce un identificativo per indicare l'account che desidera utilizzare e inserisce le credenziali di accesso per l'account.

Queste vengono convalidate rispetto alle credenziali precedentemente registrate durante la fase di provisioning dell'account.

Le credenziali possono riguardare qualcosa che l'utente conosce, qualcosa che l'utente possiede e/o qualcosa che l'utente è.

Il nome utente indica l'account che l'utente desidera utilizzare e la conoscenza della password dimostra il suo diritto a utilizzare l'account. 

\subsection{ACCESS POLICY ENFORCEMENT}

L'autorizzazione specifica ciò che un utente o un'entità può fare e l'applicazione dei criteri di accesso verifica che le azioni richieste da un utente siano consentite dai privilegi che è stato autorizzato a utilizzare.

Per assicurarsi che le azioni intraprese dall'utente siano consentite dai privilegi privilegi che gli sono stati concessi

Un'applicazione potrebbe visualizzare un messaggio che indica che un utente
non è autorizzato a visualizzare un particolare servizio.

\subsection{Sessioni}

Alcune applicazioni, in genere le applicazioni Web tradizionali e le applicazioni sensibili, consentono a un utente di rimanere attivo solo per un periodo di tempo limitato prima di richiedere all'utente di autenticarsi nuovamente.
(Una sessione tiene traccia delle informazioni)

Le impostazioni di timeout della sessione variano in genere in base alla sensibilità dei dati dell'applicazione.

\subsection{Single Sign-on}

Dopo aver effettuato l'accesso a un'applicazione, l'utente potrebbe voler effettuare un'altra operazione con un'altra applicazione. 

Il single sign-on (SSO) è la possibilità di effettuare il login una volta e poi accedere ad altre risorse o applicazioni protette con gli stessi requisiti di autenticazione, senza dover reinserire le credenziali.

Il single sign-on è possibile quando un insieme di applicazioni ha delegato l'autenticazione alla stessa entità. 


\subsection{STRONGER AUTHENTICATION}

L'autenticazione step-up è l'atto di elevare una sessione di autenticazione esistente a un livello di garanzia più elevato mediante
autenticazione con una forma di autenticazione più forte.

Ad esempio, un utente potrebbe inizialmente accedere con un nome utente e una password per stabilire una sessione di autenticazione.

In seguito, quando accede a una funzione o a un'applicazione più sensibile con requisiti di autenticazione più elevati, all'utente vengono richieste ulteriori credenziali, ad esempio una password una tantum generata sul suo telefono cellulare. 


\subsection{Logout}
Come minimo, l'atto di disconnettersi dovrebbe terminare la sessione dell'applicazione dell'utente.

Se l'utente ritorna all'applicazione, dovrà autenticarsi nuovamente prima di poter accedere.

In situazioni in cui si utilizza il single sign-on, potrebbero esserci più sessioni da terminare.

È una decisione di progettazione decidere quali sessioni debbano essere terminare quando l'utente esce da un'applicazione. 

\subsection{ACCOUNT MANAGEMENT AND RECOVERY}

Nel corso della vita di un'identità, può essere necessario modificare vari attributi del profilo utente dell'identità.

Ad esempio, un utente potrebbe dover aggiornare il proprio indirizzo e-mail o numero di telefono, la password e il nome.
In un'azienda, il profilo di un utente può essere aggiornato per riflettere una nuova posizione, un nuovo indirizzo o nuovi privilegi come i ruoli. 

Il recupero dell'account è un meccanismo per convalidare che un utente sia il legittimo proprietario di un account attraverso alcuni mezzi secondari e quindi consentire all'utente di stabilire nuove credenziali.

Ripristino della password smarrita via e-mail

\subsection{DEPROVISIONING}

Può capitare che sia necessario chiudere un account.

In questo caso, l'account dell'utente e le informazioni di identità associate devono essere deprovisionate in modo che non possano più essere utilizzate.

La deprovisioning può consistere nell'eliminazione completa dell'account e delle informazioni di identità associate o nella semplice disabilitazione dell'account, per conservare le informazioni a fini di revisione.