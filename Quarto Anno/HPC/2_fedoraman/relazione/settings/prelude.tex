\documentclass[answers, 12 pt]{exam}
%pacchetti matematica
\usepackage{amsmath}
\usepackage{amsthm}
\usepackage{amsfonts}
\usepackage{amssymb}
\usepackage{mathrsfs}

\renewcommand{\qedsymbol}{$\blacksquare$}


\renewcommand{\solutiontitle}
{\noindent\textbf{\large Soluzione}\enspace} 

%sezioni in soluzione es
\newcommand{\mysection}[1]% #1 = title
{\stepcounter{section}%
\setcounter{question}{0}%
\fullwidth{\smallskip\textbf{\normalsize #1}}}

\usepackage[italian]{babel}
\usepackage{graphicx}
\usepackage[utf8]{inputenc}


\usepackage[table,xcdraw]{xcolor}
\usepackage{setspace}
\usepackage{amssymb}

%mappe di karnaugh
\usepackage{karnaugh-map}

%disegno del circuito
\usepackage{circuitikz}


\usepackage{xcolor}
\usepackage{listings}
\colorlet{mycoolgray}{gray!40}

\lstdefinestyle{output}{
	numbers=none, % where to put the line-numbers
	numberstyle=\tiny, % the size of the fonts that are used for the line-numbers     
	backgroundcolor=\color{darkgray},
	basicstyle=\ttfamily\color{white},
	captionpos=b, % sets the caption-position to bottom
	breaklines=true, % sets automatic line breaking
	breakatwhitespace=false, 
}

\lstdefinestyle{cmd}{
	numbers=none, % where to put the line-numbers
	numberstyle=\tiny, % the size of the fonts that are used for the line-numbers     
	backgroundcolor=\color{mycoolgray},
	basicstyle=\ttfamily\color{black},
	captionpos=b, % sets the caption-position to bottom
	breaklines=true, % sets automatic line breaking
	breakatwhitespace=false, 
}

%\lstset{ 
%	numbers=none, % where to put the line-numbers
%	numberstyle=\tiny, % the size of the fonts that are used for the line-numbers     
%	backgroundcolor=\color{darkgray},
%	basicstyle=\ttfamily\color{white},
%	captionpos=b, % sets the caption-position to bottom
%	breaklines=true, % sets automatic line breaking
%	breakatwhitespace=false, 
%}


